\documentclass{article}
\usepackage{syntax}
\usepackage{varioref}
\usepackage{listings}
\begin{document}
\newcommand{\code}[1]{{\texttt{#1}}}
\newcommand{\column}[1]{{\texttt{#1}}}
\newcommand{\nonterm}[1]{{\textit{#1}}}
\newcommand{\SoQL}{SoQL}
\newcommand{\SODA}{SODA}
\newcommand{\caret}{\char`\^}
\newcommand{\transformsTo}{ $\Longrightarrow$ }
\setlength{\grammarindent}{5em}

\lstset{
    basicstyle=\ttfamily,
    keywordstyle=\ttfamily,
    identifierstyle=\ttfamily,
    stringstyle=\ttfamily,
    showstringspaces=false}

\title{\SoQL{} Level 0}
\maketitle
\section{Introduction to \SoQL{}}

\SoQL{} is a SQL-like language for querying datasets stored in a
\SODA{} provider.

TODO
\begin{itemize}
\item Dataset context
\item Simple examples?
\end{itemize}

\SoQL{} Level~0 contains operations that occur on a single dataset
without joins.

\section{Reserved words}

\SoQL{} reserves a number of identifiers for its own use.  They are
listed in figure~\ref{reservedwords} and~\ref{futurewords}.  In order
to refer to a column whose name is the same as one of these reserved
words, it must be enclosed in backticks.  For example, ``\code{SELECT
  `select` WHERE `limit` = 5}'' would select the column named
\column{select} on rows where the \column{limit} column contains the
value 5.  Without the backticks, this query would have been rejected
with a syntax error.

The reserved words in figure~\ref{futurewords} are not used in \SoQL{}
Level~0, but are still reserved for compatibility with higher levels.

\begin{figure}
\begin{center}
\begin{tabular}{l l l l l l}
AND   & ASC    & BETWEEN & BY    & DESC   & EXCEPT \\
FIRST & GROUP  & IN      & IS    & LAST   & LIMIT  \\
NOT   & NULL   & OFFSET  & OR    & ORDER & SELECT  \\
WHERE                                              \\
\end{tabular}
\end{center}
\caption{Reserved words}
\label{reservedwords}
\end{figure}

\begin{figure}
\begin{center}
\begin{tabular}{l l l l l l}
DISTINCT & FROM & FULL  & INNER & JOIN \\
LEFT     & ON   & OUTER & RIGHT        \\
\end{tabular}
\end{center}
\caption{Not presently used, but still reserved}
\label{futurewords}
\end{figure}

\section{Lexical structure}

\subsection{Identifiers}
Identifiers are not case-sensitive.  More precisely, they are
converted to lower case in the locale specified by the dataset
context\footnote{With one addition: dashes are converted to
  underscores as part of the case conversion process.}.  An
identifier matches the regex
\begin{lstlisting}
[:jidentifierstart:][:jidentifiercont:]*
\end{lstlisting}
where the two character-classes represent the set of unicode
characters accepted by the Java functions
\lstinline|isJavaIdentifierStart| and \lstinline|isJavaIdentifierPart|
on the \lstinline|java.lang.Character| class.

Identifiers can also be quoted with backticks.  Quoting is primarily
concerned with allowing keywords to be used as identifiers, but also
allows dashes within the identifier.  Some examples:
\begin{itemize}
\item \lstinline!`hello`! --- the same as \lstinline!hello!.
\item \lstinline!`hello-world`! --- legal, and the same (after case-canonicalization) as \lstinline!hello_world!.
\item \lstinline!`-hello`! --- legal, and the same (after case-canonicalization) as \lstinline!_hello!.
\item \lstinline!`where`! --- legal; an identifier, not a keyword.
\item \lstinline!`hello world`! --- illegal; spaces not allowed in identifiers.
\end{itemize}

System identifiers are names in the column namespace which are
provided by the system; names in this format cannot be introduced by
users.  They begin with a colon; implementation-specific system identifiers
begin with a colon followed by an at-sign.  They match the regex
\begin{lstlisting}
:@?[:jidentifiercont:]+
\end{lstlisting}
If backtick-quoted, system identifiers may also contain dashes.

\subsection{Numbers}
Number literals are anything which matches the regular expression
\begin{lstlisting}
[0-9]+(\.[0-9]*)?([eE][+-]?[0-9]+)?
\end{lstlisting}
Integer literals are any number literal which does not contain a
decimal point or an exponent.  They may be used anywhere a number is
allowed, but some places in the grammar expect specifically a literal
integer.

\subsection{Strings}
String literals come in two forms: Java-style strings and SQL-style
strings.

Java-style strings are delimited with double quotation marks and use
backslash as an escape character.  They may not contain newlines but
otherwise no characters are off-limits.  Figure~\vref{escapes} shows
the legal escape sequences.  Any other character following a backslash
is an error.
\begin{figure}
\begin{center}
\begin{tabular}{c l}
\lstinline|\b| & Backspace (code point 8) \\
\lstinline|\f| & Formfeed (code point 12) \\
\lstinline|\n| & Newline (code point 10) \\
\lstinline|\r| & Carriage return (code point 13) \\
\lstinline|\t| & Tab (code point 9) \\
\lstinline|\u| & Unicode (followed by four hex digits representing the code point) \\
\lstinline|\U| & Unicode (followed by six hex digits representing the code point) \\
\lstinline|\"| & Double-quote \\
\lstinline|\\| & Backslash \\
\end{tabular}
\end{center}
\caption{Legal escape sequences}
\label{escapes}
\end{figure}

SQL-style strings are delimited with single quotation marks.  They
have no escape character and may contain anything, including newlines.
To include a single-quote in a SQL-style string, double it.

\subsection{Comments}

\SoQL{} supports comments initiated by a double-hyphen
(``\lstinline|--|'') and terminated by the end of the line.

\section{Grammatical structure}

\subsection{Select}
\begin{grammar}
<select> ::= `SELECT' <select-list> [<where-clause>] [<group-by-clause> [<having-clause>]] [<order-by-clause>] [<limit-clause>] [<offset-clause>]
\end{grammar}

No \code{FROM} clause is necessary because it is implicit in the
dataset context.

\subsubsection{Selection lists}
\begin{grammar}
<select-list> ::= <system-star> [`,' <only-user-star-select-list>]
 \alt <only-user-star-select-list>

<only-user-star-select-list> ::= <user-star> [`,' <expression-select-list>]
 \alt <expression-select-list>

<expression-select-list> ::= <selection> (`,' <selection>)*

<system-star> ::= `:*' [`(' `EXCEPT' system-identifier (`,' system-identifier)* `)']

<user-star> ::= `*' [`(' `EXCEPT' user-identifier (`,' user-identifier)* `)']

<selection> ::= <expression> [`AS' user-identifier]
\end{grammar}

Unlike SQL, \SoQL{} does not automatically include \emph{all} columns
when \lstinline|*| is used in the selection.  Instead, only the
dataset's user-defined columns are included.  System columns may be
included individually by name or through use of the \lstinline|:*|
token\footnote{It is meant to suggest ``all names which begin with a
  colon'' but also includes implementation-specific system columns.}.

Note that this grammar implies that \lstinline|:*| and \lstinline|*|,
if they appear at all, must be at the start of the selection list and,
if both are present, in that order.

\subsubsection{\code{WHERE} clause}
\begin{grammar}
<where-clause> ::= `WHERE' <expression>
\end{grammar}

A \code{WHERE} clause filters the input table to those rows for which
the given \nonterm{expression} evaluates to true.

\subsubsection{\code{GROUP BY} clause}
\begin{grammar}
<group-by-clause> ::= `GROUP' `BY' <expression-list>

<expression-list> ::= <expression> (`,' <expression>)*
\end{grammar}

\subsubsection{\code{HAVING} clause}
\begin{grammar}
<having-clause> ::= `HAVING' <expression>
\end{grammar}

A \code{HAVING} clause filters the grouped table to those rows for
which the given \nonterm{expression} evaluates to true.  If there is
no \code{GROUP BY} clause, a \code{HAVING} clause is an error.

\subsubsection{\code{ORDER BY} clause}
\begin{grammar}
<order-by-clause> ::= `ORDER' `BY' <ordering-list>

<ordering-list> ::= <ordering> (`,' <ordering>)*

<ordering> ::= <expression> [`ASC' | `DESC'] [`NULL' (`FIRST' | `LAST')]
\end{grammar}

\code{ORDER BY} sorts the output table by some criteria.

If unspecified, the default order is ``sort ascending'' with NULL
values placed in an implementation-defined place.

\subsubsection{\code{LIMIT} clause}
\begin{grammar}
<limit-clause> ::= `LIMIT' integer-literal
\end{grammar}

\subsubsection{\code{OFFSET} clause}
\begin{grammar}
<offset-clause> ::= `OFFSET' integer-literal
\end{grammar}

\subsection{Expressions}

Figure~\vref{precedence} contains a list of precedence levels which
may be easier to read than the formal grammar.

\begin{grammar}
<identifier> ::= user-identifier | system-identifier

<identifier-or-funcall> ::= <identifier> [`(' <expression-list> | `*' `)' ]
\end{grammar}

An identifier in this production can be in one of two forms---either
it can be an alias for an expression or a reference to a column (i.e.,
it refers to a value), or it can be the name of a function.  Value
identifiers are case-canonicalized in the locale of the dataset
context.  Function names are case-canonicalized in the locale of the
\SoQL{} implementation.

\begin{grammar}
<literal> ::= number-literal | string-literal | boolean-literal | `NULL'

<value> ::= <identifier-or-funcall> | <literal> | `(' <expression> `)'

<dereference> ::= <dereference> `.' <identifier>
 \alt <dereference> `[' <expression> `]'
 \alt <value>
\end{grammar}

The dot-form of a dereference is syntactic sugar for a
bracket-dereference with a literal string.  As a result, this form is
not case-canonicalized at all.

\begin{grammar}
<cast> ::= <cast> `::' <identifier> | <dereference>
\end{grammar}

The identifier in a cast is part of the \SoQL{} implementation, and
therefor it is case-canonicalized in that locale.

\begin{grammar}
<unary> ::= [`+' | `-'] <unary> | <cast>

<exp> ::= <unary> [`\caret' <exp>]
\end{grammar}

Unlike all other binary operators, exponentiation is right-associative.

\begin{grammar}
<factor> ::= [<factor> (`*' | `/' | '\%')] <exp>

<term> ::= [<term> (`+' | `-' | `||')] <factor>

<order> ::= [<order> (`=' | `==' | `!=' | `<>' | `<' | `<=' | `=>' | `>')] <term>

<is-between-in> ::= <is-between-in> `IS' [`NOT'] `NULL'
 \alt <is-between-in> [`NOT'] `BETWEEN' <is-between-in> `AND' <is-between-in>
 \alt <is-between-in> [`NOT'] `IN' `(' <expression>  [ `,' <expression> ]* `)'
 \alt <order>

<negation> ::= `NOT' <negation>
 \alt <is-between-in>

<conjunction> ::= [<conjunction> `AND'] <negation>

<disjunction> ::= [<disjunction> `OR'] <conjunction>

<expression> ::= <disjunction>
\end{grammar}

\begin{figure}
\begin{center}
\begin{tabular}{c}
\lstinline|()| (function application) \\
\lstinline|. []| (field/subscript selection) \\
\lstinline|::| \\
\lstinline|+ -| (unary) \\
\lstinline|^| \\
\lstinline|* / \%| \\
\lstinline!+ - ||! \\
\lstinline|= == != <> < <= > >=| \\
\lstinline|IS (NOT) NULL, BETWEEN| \\
\lstinline|NOT| \\
\lstinline|AND| \\
\lstinline|OR| \\
\end{tabular}
\end{center}
\caption{Precedence levels from highest to lowest}
\label{precedence}
\end{figure}

\section{Identifiers and scope}

Aliases introduce new names; they do not actually rename things.  In a
statement with the selection \lstinline|a AS b|, it is still legal to
refer to \code{a}; it simply will not occur in the output columns
unless separately selected as itself.

The full algorithm for naming things is:
\begin{enumerate}
\item Expressions which are simple column-names, or the columns
  imported by \code{*} or \code{:*}, are considered to be
  semi-explicitly aliased to themselves.  All other columns are in
  scope but are unaliased.
\item It is an error for any pair of semi-explicitly or explicitly
  aliased expressions to have the same alias.  Explicitly aliased
  expressions can hide columns that exist on the dataset but are not
  referred to in the selection list.
\item It is an error for an explicitly aliased identifier to refer to
  itself, directly or indirectly.
\item After semi-explicit and explicit aliases are assigned, the
  complex expressions have their aliases assigned according to the
  paragraph below.  Note that simply wrapping a simple expression in
  parenthesis DOES make it a complex expression (and hence
  ``\code{(a)}'' will generate a name like \lstinline|a_1| to avoid
  conflicting with column \lstinline|a|).  This process will never
  hide any existing column or alias, which means that dataset schema
  changes can cause these implicit aliases to change!
\end{enumerate}

Expressions more complex than a simple column-name but without
explicit aliases have an implicit alias automatically assigned in
left-to-right order across the selection list.  This alias consists of
all the \code{identifier}s, \code{literal}, and keywords\footnote{The
  ``\code{-}'' operator is not a keyword, even though a single dash is
  a valid identifier when quoted.} in the expression, in the order in
which they appear in the expression, transformed in the following
manner:
\begin{enumerate}
\item All non-\code{javaIdentifierPart}-or-dash characters are
  replaced by SYNTHETIC underscores (a special marker which is
  distinguishable from all other characters, including underscores).
\item The tokens are joined with SYNTHETIC underscores into a single
  unit.
\item Runs of SYNTHETIC underscores are collapsed.
\item SYNTHETIC underscores at the start, end, or adjacent to a real
  underscore or dash are removed.
\item All SYNTHETIC underscores are replaced by real underscores.
\item If the resulting string does not start with a dash or a
  \code{javaIdentifierStart}, a single underscore is prefixed.
\end{enumerate}
Finally, if this generated identifier is the same as any existing
column names or aliases (including previously generated synthetic
aliases), the first positive integer is found such that suffixing it
(separating by an underscore if the generated alias does not already
end with a dash or underscore) causes no conflicts.

Examples:
\begin{itemize}
\item \lstinline|select a, a| is an error --- two semi-explicitly
  aliased expressions have the same alias.
\item \lstinline|select a, (a)| is not an error.  The second column
  will be implicitly aliased to \lstinline|a_1| (assuming there is no
  exisiting \lstinline|a_1| column in the dataset context).
\item \lstinline|select a as a| is an error because ``a'' is
  explicitly defined in terms of itself.
\item \lstinline|select count_a as a, count(a)| is also an error
  because ``a'' is indirectly defined in terms of itself.
\item \lstinline|select 1 + num| is not an error.  The selected
  column will be implicitly aliased to \lstinline|_1_num| if there
  is no such column in the dataset, or \lstinline|_1_num_1| if there is.
\item \lstinline|select a as b| is not an error, even if there is a
  column \code{b} on the dataset already.  If there is, column
  \code{b} is hidden by the alias.
\item \lstinline|select *, a as b| is only an error if there is a
  \code{b} column on the dataset, because it is a conflict between a
  semi-explicitly aliased column and an explicitly aliased one.
\item \lstinline|select * (except b), a as b| is not an error.  Column
  \code{b} is hidden by the alias.
\item \lstinline|select a + __b__| will have its column aliased to
  \lstinline|a__b__|, or (in the event that name is in use by a
  column) \lstinline|a__b__1|.  Note that no underscores are
  introduced in either case.
\item \lstinline|select (a),(a),(a)| will generate aliases
  \lstinline|a_1|, \lstinline|a_2|, and \lstinline|a_3|.
\item \lstinline|select :id - 1| will generate alias \lstinline|id_1|.
\end{itemize}

\section{Grouping}

Usually, expressions are evaluated in a context referred to as
``ungrouped'', where each expression refers to a single value.  A
second execution context, the ``grouped'' context, is introduced by
either using an aggregation function in the selection list or
\code{ORDER BY} clause, or by the presence of a \code{GROUP BY} or
\code{HAVING} clause.  In this context, expressions refer to
collections of values.  Only literals, expressions which are
structurally equal (see section~\vref{structuralequality}) to
expressions which appear in the \code{GROUP BY} clause, applications
of aggregation functions, and expressions built from these categories
are valid in the grouped contex.  Within the \code{WHERE} clause and
the argument list of aggregation functions, the ungrouped context is
used and aggregation functions may not be.

\begin{itemize}
\item \lstinline|select x as a where a > 3| is legal.
\item \lstinline|select x as a where a > 3 group by x| is legal---the
  alias expands to an expression which is legal in an ungrouped context.
\item \lstinline|select x, sum(y) as total where total > 3 group by x| is
  illegal because the alias expands to an aggregate function.
\item \lstinline|select x, y group by x| is illegal because the ``y''
  is not structurally equal to one of the \code{GROUP BY} expressions.
\item \lstinline|select x + y group by x + y| is legal because the selected
  expression is in the \code{GROUP BY} list.
\end{itemize}

\section{Typechecking}

Typechecking proceeds in several phases.  First, the selection list is
sorted to provide a typechecking order in which any aliases referred
to by an expression are typechecked before it.  Each is typechecked in
this order, making the earlier aliases available to later ones.  After
all aliases are typechecked, all other expressions are
typechecked\footnote{No particular order is required.} with the typed
aliases also in scope.

Within a single expression, typechecking proceeds depending on what
kind of expression is seen.  Expressions fall into the following
categories: literal, identifier, function call, cast.  All operators
except \code{::} are considered to be ordinary function calls.

\subsection{Function calls}

Functions may be overloaded.  In order to typecheck a function call,
the set of functions with the same name and arity of the expression
under consideration is found.  Each candidate function is examined in
turn.  For each parameter, if the exact type of the corresponding
argument cannot be passed in without conversion, then if an implicit
conversion is available it is applied to the argument.  Otherwise, the
function is incompatible with the argument list and is discarded.

At the end of the whole process of looking up implicit conversions and
discarding functions which cannot be called, if there is a single
function with the fewest implicit conversions applied, that function
is the one selected and the type of the whole expression is the type
of the resulting function.  Otherwise, if there is a function candidate
with fewer distinct types in its parameter list and result, that candidate
is chosen.  Otherwise, the call is ambiguous and typechecking fails.

\subsection{Literals}

Literals may be subject to a greater degree of implicit conversion
than their parent type.  The typechecker is allowed to use the values
of literals to provide more options for implicit conversions---for
example, allowing string literals which look like dates to have a
``\code{to_date}'' function applied to them when used in a context
that expects a date.  An alias is never considered to have a literal
type, even when it refers to a literal expression.

\subsection{Identifiers}

Identifiers are resolved by first looking the indentifier up in the
set of aliases.  If an alias with the same name is defined,
typechecking is finished (since it may use the result of the previous
typechecking pass without modification).  Otherwise, it is looked up
in the dataset context's set of columns.  If it is found there, the
type is the type of the found column.  Otherwise, it is an error.

\subsection{Casts}

Casts are syntactic sugar for ordinary function calls with the
exception that implicit conversions are not consulted in order to find
a valid conversion function.  For example, given types A, B, and C,
with A implicitly convertable to B and a cast function \code{b_to_c}
from B to C, \lstinline|a :: C| would fail as uncastable, but
\lstinline|b :: C| would succeed and result in the expression
\lstinline|b_to_c(b)|.  Whereas explicitly typechecking
\lstinline|b_to_c(a)| would find and apply the implicit conversion
from \code{A} to \code{B}.

\section{Structure equality}\label{structuralequality}

Two expressions are considered structurally equal if their ``normal
form'' representations are the same.

\SoQL{} normal form is a notional alternate syntax for \SoQL{} which
defines an unambiguous textual representation of a \SoQL{} expression.
Normal Form \SoQL{} exists only for well-typed expressions and
consists only of literals, column references, and function calls.  A
\SoQL{} query is transformed in the following manner:
\begin{enumerate}
\item All comments are replaced by whitespace.
\item All aliases are expanded into their source expressions.
\item All identifiers except for those used as the ``argument'' to the
  \lit{.} operator are case-canonicalized (and not backtick-quoted).
\item All string literals are rendered as SQL-style strings.
\item Operators except for \lit{.} are translated into function calls
  by placing the operator before a parameter list as if it were a
  normal function, preceded by \code{op\$}.  \lit{.} is desugared into
  an application of \code{op\$[]}.
\item \code{\synt{expression} IS NULL} is translated into
  \code{\#is_null(\synt{expression})}.
\item \code{\synt{expression} IS NOT NULL} is translated into
  \code{\#is_not_null(\synt{expression})}.
\item \code{\synt{e1} BETWEEN \synt{e2} AND \synt{e3}} is translated
  into \code{\#between(\synt{e1},\synt{e2},\synt{e3})}.
\item \code{\synt{e1} NOT BETWEEN \synt{e2} AND \synt{e3}} is translated
  into \code{\#not_between(\synt{e1},\synt{e2},\synt{e3})}.
\item \code{\synt{e} IN \synt{e2} (\synt{es...})} is translated
  into \code{\#in(\synt{e1},\synt{es...})}.
\item \code{\synt{e} NOT IN \synt{e2} (\synt{es...})} is translated
  into \code{\#not_in(\synt{e1},\synt{es...})}.
\item All parentheses that do not denote function application are removed.
\item All whitespace outside string literals is removed.
\end{enumerate}

Examples:
\begin{itemize}
\item \code{1+1}\transformsTo\code{op\$+(1,1)}
\item \code{`where` between 1 and 10}\transformsTo\code{\#between(where,1,10)}
\item \code{(a or b) and c}\transformsTo\code{op\$and(op\$or(a,b),c)}
\item \code{a or b and c}\transformsTo\code{op\$or(a,op\$and(b,c))}
\item \code{count(*)}\transformsTo\code{count(*)}
\item \code{a[foo]}\transformsTo\code{op\$[](a,foo)}
\item \code{a.Foo}\transformsTo\code{op\$[](a,'Foo')}
\end{itemize}

\end{document}
